%
%----------------------------------------------------------------------------------------
%	PACKAGES AND DOCUMENT CONFIGURATIONS
%----------------------------------------------------------------------------------------
%

\documentclass[12pt]{article} %report, article, amsart, exam
\usepackage[letterpaper,margin=1in]{geometry}
\usepackage{fancyhdr,color}
\usepackage{tikz,graphicx,multicol}
\usepackage{amssymb,euscript,eufrak,nicefrac,enumitem}
\usepackage{amsfonts,amsmath,amsthm} %don't need with 'amsart' document class
\usepackage{hyperref}
\usepackage{comment}
\usepackage{scrextend} %needed for addmargin environment
\usepackage{graphicx} 
\usepackage{listings}

\usepackage{color}
\usepackage{accsupp}
\usepackage{booktabs}
\usepackage{subcaption}

\definecolor{dkblue}{rgb}{0,0,0.5}
\definecolor{comment}{rgb}{1,0,0}
\definecolor{mauve}{rgb}{.627,.126,.941}
\definecolor{purple}{rgb}{0.5, 0, 0.545098}

\lstdefinestyle{customc}{
  belowcaptionskip=1\baselineskip,
  breaklines=true,
  frame=L,
  xleftmargin=\parindent,
  language=C,
  showstringspaces=false,
  basicstyle=\footnotesize\ttfamily,
  keywordstyle=\bfseries\color{green!40!black},
  commentstyle=\itshape\color{purple!40!black},
  identifierstyle=\color{blue},
  stringstyle=\color{orange},
}

\lstdefinestyle{customasm}{
  belowcaptionskip=1\baselineskip,
  frame=L,
  xleftmargin=\parindent,
  language=[x86masm]Assembler,
  basicstyle=\footnotesize\ttfamily,
  commentstyle=\itshape\color{purple!40!black},
}

\lstset{escapechar=@,style=customc}

%
%----------------------------------------------------------------------------------------
%% Define headers & footers
%----------------------------------------------------------------------------------------

\pagestyle{fancy}
   \lhead{} 
   %\chead{Loyola University Chicago} 
   \rhead{}
   \renewcommand{\headrulewidth}{0pt}
   \addtolength{\footnotesep}{5mm}
 
%
%----------------------------------------------------------------------------------------
%% Some user-defined colors
%----------------------------------------------------------------

\setlength{\parskip}{1em}
\renewcommand{\baselinestretch}{1.3}

%
%----------------------------------------------------------------------------------------
%% BEGIN: topmatter
%----------------------------------------------------------------------------------------
%

%	1) Project title
%
%	2) What algorithm are you targeting and why?
%
%	- 2-3 sentences (or bullets) will suffice but more is always welcome. Tell us what the algorithm does and why you selected this for optimization / 
%	parallelization.
%
%	3) How do you plan to improve the performance through parallelism?
%
%	- 2-3 sentences (or bullets), again more is good. Tell us what type of parallelism (and optimization) you'll implement and how. This can be data (vector), 
%	thread, or distributed memory parallelism. Memory efficiency, though not parallelism, can be another tool to use. You need at least 1 form of 
%	parallelism.

%??Data: Weekly Raw Market Transactions in milliseconds - (FIX/FAST - https://en.wikipedia.org/wiki/FAST_protocol) ~ 30M transactions
%Problem: Process raw transactional data (daily, hour, minute, second - volume), (filtering - map, reduce) in particular during month/weeks before and during the fast crash of 2010 ( https://en.wikipedia.org/wiki/2010_Flash_Crash )



\title{COMP 464 - High Performance Computing \\ High Frequency Trading \& 2010 Market Flash Crash} % Title
\author{
Loyola University Chicago \\
Jose Luis Rodriguez 
} % Author name
\date{\today} % Date for the report

%
%% END: topmatter
%%------------------------------

\begin{document}

\maketitle

\thispagestyle{fancy}

%---------------------------------------------------------------------------------------- 
%	SECTION 1 - PROBLEM STATEMENT
%----------------------------------------------------------------------------------------

\section{Overview}

With the advent of low cost advance computational capabilities and rapid grow of market transactions. The ability to process a high volume of transactions and calculation in an ever decreasing window of time have become a corner stone of modern financial markets (cite). The months leading to May of 2010 in the U.S were filled with political unrest overseas and news about the European debt crisis, then in May 6, 2016 protections for the default of the Greek government sovereign debt increased and the Euro declined sharply against the U.S Dollar and Japanese Yen. 

The turmoil of the afternoon of May 6, 2010 in futures and securities markets were down 4\% from previous day, then prices of futures contracts in particular stock index instruments declined suddenly another 5\%-6\% in very short period of time before rebounding within a very short period of time. The complexities and some of the details of what happened that day are covered on a report to the U.S Senate Committee on Banking and are beyond the scope of this project. (cite)

\newpage

As computational capabilities continue to grow exponentially and financial markets around the world are increasingly dependable on automated systems it is worth to spend some time analysis the market depth data of the weeks leading to and the day of May 6, 2010 to have a better understanding of market behavior during this high volatility period and to compare some of this project findings with official report to congress from the CFTC and SEC. The data used in this project is from CME Group and corresponds to market depth transactions in milliseconds of the E-Mini S\&P 500 futures and options contracts. (cite)

%??Data: Weekly Raw Market Transactions in milliseconds - (FIX/FAST - https://en.wikipedia.org/wiki/FAST_protocol) ~ 30M transactions
%Problem: Process raw transactional data (daily, hour, minute, second - volume), (filtering - map, reduce) in particular during month/weeks before and during the fast crash of 2010 ( https://en.wikipedia.org/wiki/2010_Flash_Crash )

%----------------------------------------------------------------
%	SECTION 2 - METHODOLOGY
%----------------------------------------------------------------

\section{Methodology}

The market depth data that the CME Group provides contains all market data messages required to recreate the order book (list of orders that a trading firm uses to record the interest of buyers and sellers in a particular financial instrument.) each message contains between five to ten orders deep in futures markets and three orders deep in options markets this data is time stamped to the millisecond allowing for an in depth analysis of the price movement. (cite)

In order to process the large volume of transactions (millions of transactions per week) and the goal is to compute daily, hourly, minute, seconds, millisecond volume and other data metrics it is necessary to implement statistical operations such as distributions, average in parallel as well as filtering map-reduce type jobs are ideal for type of task as the nature of the data (independent transactions) allows for parallel processing in most cases.

%----------------------------------------------------------------
%	SECTION 3 - PERFORMANCE
%----------------------------------------------------------------

\section{Performance}

When processing the market depth data there different factors to consider such as the distribution of the files amount different nodes (Hadoop) and how to distributed evenly the work that each node would perform. To measure performance  scalability metrics can be use in each of the statistical and filtering operations.

\newpage
%----------------------------------------------------------------
%	SECTION 4 - REFERENCE
%----------------------------------------------------------------

\section{Reference}

\begin{flushleft}
Rustler User Guide -- Hadoop Cluster \url{https://portal.tacc.utexas.edu/user-guides/rustler}

Introduction to High Performance Scientific Computing -- Victor Eijkhout \\ \url{http://pages.tacc.utexas.edu/~eijkhout/istc/istc.html}

Kirilenko, Andrei; Kyle, Albert S.; Samadi, Mehrdad; Tuzun, Tugkan (May 5, 2014), The Flash Crash: The Impact of High Frequency Trading on an Electronic Market (PDF), retrieved 8 November 2017 \\ \url{http://www.cftc.gov/idc/groups/public/@economicanalysis/documents/file/oce_flashcrash0314.pdf}

U.S. Securities and Exchange Commission and the Commodity Futures Trading Commission (September 30, 2010). "Findings Regarding the Market Events of May 6, 2010" \\ \url{https://www.sec.gov/news/studies/2010/marketevents-report.pdf}

\end{flushleft}


\end{document}